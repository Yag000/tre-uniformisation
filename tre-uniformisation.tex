\documentclass{article}
\usepackage[utf8]{inputenc}
\usepackage[T1]{fontenc}

\usepackage{amsmath}
\usepackage{amssymb} 
\usepackage{amsthm}  
\usepackage{dsfont}
\usepackage{mathrsfs}
\usepackage{mathtools}

\usepackage{tikz-cd}
\usepackage{geometry}
\usepackage{hyperref}        

\usepackage[shortlabels]{enumitem}

\usepackage{fancyhdr}

\usepackage[hyperref,paper]{knowledge}

\usepackage{tre-uniformisation}

\knowledgeconfigure{label scope=false, notion, quotation, protect quotation={tikzcd, automata}}


\knowledge{tree}{notion}
\knowledge{Deterministic Bottom-Up tree automaton}[DFTA]{notion}
\knowledge{interpretation}{notion}


\begin{document}

\title{Uniformisation de MSO sur les arbres}

\maketitle

\section{Definitions}


\begin{definition}[Tree]
	A ""tree"" over an alphabet $\Sigma$ can be recursively defined as :
	\begin{itemize}
		\item $a$, $a \in \Sigma$.
		\item $a(t,'t')$, $a \in \Sigma$, $t,t'$ trees.
	\end{itemize}
\end{definition}

\begin{definition}[""Deterministic Bottom-Up tree automaton""]
	We define a Deterministic Bottom-Up tree automaton (DFTA) as a tuple :
	$$ (\Sigma, Q, \text{init}, \delta, F) $$
	Where :
	\begin{itemize}
		\item $\Sigma$ is an alphabet
		\item $Q$ a finite set of states
		\item $\text{init} : \Sigma \to  Q$ a function that initializes the states of the leaves.
		\item $\delta: \Sigma \times Q \times Q$ is the transition function.
		\item $F \subset Q$ are the final states.
	\end{itemize}
\end{definition}

\begin{definition}[Interpretation of an automaton]
	We define the ""interpretation"" of an automaton $A$ over an alphabet $\Sigma$ as :
	\begin{eqnarray*}
		\interpret A : \tree_{\Sigma} &\to& Q \\
		\interpret A (a) &=& \text{init}_a \\
		\interpret A (a(t,t')) &=& \delta_a (\interpret A (t), \interpret A (t') )
	\end{eqnarray*}
\end{definition}

\begin{definition}[Langage of an automaton]
	Let $A$ be a "DFTA", we define it's associated language as :
	$$L_A = \setdef {t \in \tree_{\Sigma}} {\interpret A (t) \in F}$$

\end{definition}

\bibliographystyle{alpha}
\bibliography{tre-uniformisation}

\end{document}
