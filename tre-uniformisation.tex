\documentclass{article}
\usepackage[utf8]{inputenc}
\usepackage[T1]{fontenc}

\usepackage{amsmath}
\usepackage{amssymb} 
\usepackage{amsthm}  
\usepackage{dsfont}
\usepackage{mathrsfs}
\usepackage{mathtools}

\usepackage{tikz-cd}
\usepackage{geometry}
\usepackage{hyperref}        

\usepackage[shortlabels]{enumitem}

\usepackage{fancyhdr}

\usepackage[hyperref,paper]{knowledge}

\usepackage{tre-uniformisation}

\knowledgeconfigure{label scope=false, notion, quotation, protect quotation={tikzcd, automata}}


\knowledge{tree}{notion}
\knowledge{Deterministic Bottom-Up Tree Automaton}[DFTA]{notion}
\knowledge{interpretation}{notion}


\begin{document}

\title{Uniformisation de MSO sur les arbres}

\maketitle

\section{Definitions}


\begin{definition}[Tree]
	A ""tree"" over an alphabet $\Sigma$ is recursively defined as follows:
	\begin{itemize}
		\item $a$, where $a \in \Sigma$.
		\item $a(t,t')$, where $a \in \Sigma$ and $t, t'$ are trees.
	\end{itemize}
\end{definition}

\begin{definition}[""Deterministic Bottom-Up Tree Automaton""]
	A Deterministic Bottom-Up Tree Automaton (DFTA) is defined as a tuple:
	$$ (\Sigma, Q, \text{init} : \Sigma \to Q, \delta : \Sigma \times Q \times Q \to Q, F) $$
	where:
	\begin{itemize}
		\item $\Sigma$ is an alphabet.
		\item $Q$ is a finite set of states.
		\item $\text{init}$ is a function that initializes the states of the leaves.
		\item $\delta$ is the transition function.
		\item $F$ is the set of final states.
	\end{itemize}
\end{definition}


\begin{definition}[""Non-Deterministic Bottom-Up Tree Automaton""]
	A Non-Deterministic Bottom-Up Tree Automaton (NFTA) is defined as a tuple:
	$$ (\Sigma, Q, I \subseteq \Sigma \times Q, \Delta \subseteq Q \times Q \times \Sigma \times Q, F) $$
\end{definition}

\begin{definition}[Interpretation of an Automaton]
	The ""interpretation"" of an automaton $A$ over an alphabet $\Sigma$ is defined as follows:
	\begin{eqnarray*}
		\interpret A : \tree_{\Sigma} &\to& Q \\
		\interpret A (a) &=& \text{init}_a \\
		\interpret A (a(t,t')) &=& \delta_a (\interpret A (t), \interpret A (t'))
	\end{eqnarray*}
\end{definition}

\begin{definition}[Language of an Automaton]
	Let $A$ be a "DFTA". Its associated language is defined as:
	$$L_A = \setdef {t \in \tree_{\Sigma}} {\interpret A (t) \in F}$$
\end{definition}

\bibliographystyle{alpha}
\bibliography{tre-uniformisation}

\end{document}
