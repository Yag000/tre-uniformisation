\documentclass{article}
\usepackage[utf8]{inputenc}
\usepackage[T1]{fontenc}

\usepackage{amsmath}
\usepackage{amssymb} 
\usepackage{amsthm}  
\usepackage{dsfont}
\usepackage{mathrsfs}
\usepackage{mathtools}
\usepackage{stmaryrd}

\usepackage{tikz-cd}
\usepackage{geometry}
\usepackage{hyperref}        

\usepackage[shortlabels]{enumitem}

\usepackage{fancyhdr}

\usepackage[hyperref]{knowledge}

\knowledgeconfigure{label scope=false, notion, quotation, protect quotation={tikzcd, automata}}

%% Color

\usepackage{color}
\usepackage{xcolor}

\definecolor{red}{rgb}{1,0,0}
\definecolor{green}{rgb}{0,1,0}
\definecolor{darkgreen}{rgb}{0,0.5,0}
\definecolor{blue}{rgb}{0,0,1}
\definecolor{violet}{rgb}{0.4,0,0.8}

\newcommand{\red}[1]{\textcolor{red}{#1}}
\newcommand{\green}[1]{\textcolor{green}{#1}}
\newcommand{\darkgreen}[1]{\textcolor{darkgreen}{#1}}
\newcommand{\blue}[1]{\textcolor{blue}{#1}}
\newcommand{\violet}[1]{\textcolor{violet}{#1}}


\hypersetup{
	colorlinks,
	linkcolor={red!50!black},
	citecolor={blue!50!black},
	urlcolor={blue!80!black}
}

%% Math

%%% Proofs

\newcommand{\bimpRL}{\fbox{$\Leftarrow$}}
\newcommand{\bimpLR}{\fbox{$\Rightarrow$}}

\newenvironment{proofI}[1][\proofname]{%
	\begin{proof}[#1]$ $\par\nobreak\ignorespaces
		}{%
	\end{proof}
}

\newcommand{\reason}[1]{\quad\left(\textit{#1}\right)}

%%% Sets

\newcommand{\N}{\mathbb{N}}

\newcommand{\Z}{\mathbb{Z}}
\newcommand{\Zn}[1]{\mathbb{Z}/#1\mathbb{Z}}
\newcommand{\F}{\mathbb{F}}
\newcommand{\Q}{\mathbb{Q}}

\newcommand{\R}{\mathbb{R}}
\newcommand{\Rn}{\R^n}

\newcommand{\C}{\mathbb{C}}

\newcommand{\K}{\mathbb{K}}


\newcommand{\set}[1]{\left\{#1\right\}}
\newcommand{\setdef}[2]{\set{#1 \mid #2}}
\newcommand{\parts}[1]{\mathcal{P}\left(#1\right)}
\newcommand{\enum}[2]{\set{#1,\ldots,#2}}


%%% Logic

\newcommand{\ra}{\rightarrow}
\newcommand{\contradict}{\lightning}


%%% Functions

\newcommand{\restr}[2]{#1_{\restriction #2}}
\newcommand{\im}[1]{\text{Im}\left(#1\right)}

\newcommand{\pgcd}[2]{\text{pgcd}(#1,#2)}
\newcommand{\car}{\text{car}}
\newcommand{\id}{\text{id}}


%%% Polynomials

\newcommand{\QX}{\mathbb{Q}[X]}
\newcommand{\RX}{\mathbb{R}[X]}
\newcommand{\CX}{\mathbb{C}[X]}

\newcommand{\AX}{A[X]}

%%% Vector spaces

\newcommand{\sprod}[2]{\left\langle #1, #2 \right\rangle}

\newcommand{\spec}{\text{Spec}}

\newcommand{\abs}[1]{\left|#1\right|}
\newcommand{\norm}[1]{\abs{\abs{#1}}}
\newcommand{\normop}[1]{\abs{\abs{\abs{#1}}}}

\newcommand{\dist}[2]{\text{ d}\left(#1, #2\right)}
\newcommand{\vect}[1]{\text{Vect}\left( #1 \right)}

\newcommand{\Mn}[2]{\mathcal{M}_{#1}(#2)}

%%% Probability notation

\newcommand{\Pro}{\mathscr{P}}
\newcommand{\E}{\mathbb{E}}

\newcommand{\1}{\mathds{1}}

%% Abbreviations

\newcommand{\ie}{\textit{i.e.} }
\newcommand{\eg}{\textit{e.g.} }


%% Semantics

\newcommand{\syntaxHeader}[3]{#1 &::=& #2 &\quad\text{#3}\\}
\newcommand{\syntaxExtension}[1]{\syntaxHeader{#1}{\cdots}{}}
\newcommand{\syntax}[2]{&|&#1&\quad\text{#2}\\}

\newenvironment{syntaxdef} {\begin{mathpar} \begin{array}{lcll}} {\end{array} \end{mathpar}}


%% Misc commands

\newcommand{\quotes}[1] {"#1"}
\newcommand{\todo}[1]{\red{TODO\ifx#1~ \else: #1\fi}}

%% Environments

\theoremstyle{plain}
\newtheorem{theorem}{Theorem}[section]
\newtheorem{coro}[theorem]{Corollary}
\newtheorem{lemma}[theorem]{Lemma}
\newtheorem{prop}[theorem]{Proposition}

\theoremstyle{definition}
\newtheorem{definition}[theorem]{Definition}
\newtheorem{notation}[theorem]{Notation}
\newtheorem{example}{Example}[subsection]
\newtheorem{exercice}{Exercice}[subsection]


\theoremstyle{plain}
\newtheorem{remark}{Remark}[subsection]



%% Project specific


\newcommand{\app}[1]{\ifx#1~ \else \left( #1 \right)\fi}

\newcommand{\tree}{\text{\kl[tree]{Tree}}}


\knowledgenewrobustcmd \nodes [1] {\cmdkl {\text{Nodes}} \app {#1}}
\knowledgenewrobustcmd \leaves [1] {\cmdkl {\text{Leaves}} \app {#1}}

\knowledgenewrobustcmd \tlabel [2] {\cmdkl {#1} \left( #2\right)}

\knowledgenewrobustcmd \exec {\cmdkl {\rho}}

\knowledgenewrobustcmd \interpret [2] {\cmdkl{\llbracket} #1 \cmdkl{\rrbracket} \app{#2}}

\knowledgenewrobustcmd{\ancestor}{\mathrel{\cmdkl{\sqsubseteq}}}

\knowledgenewrobustcmd \Model [1] {\cmdkl{\mathcal M} \app{#1}}

\knowledgenewrobustcmd \Det [1] {\cmdkl{\text{Det}} \app{#1}}

\knowledgenewrobustcmd \init {\cmdkl {\text{init}}}


\knowledgenewrobustcmd \leaf {\cmdkl {\text{leaf}}}
\knowledgenewrobustcmd \treeroot {\cmdkl {\text{root}}}
\knowledgenewrobustcmd \children {\cmdkl {\text{children}}}


\knowledgenewrobustcmd \Atimes {\cmdkl {\times}}

\newcommand{\setcomp}[1]{{#1}^\complement}

\newcommand{\La}{\mathcal{L}}
\knowledgenewrobustcmd \lang [1] {\cmdkl{\La} \app{#1}}

\newcommand{\decomp}[2] {
	#1_1 #1_2 \ldots #1_{#2}
}

\newcommand{\motDecomp}[2] {
	#1 = \decomp{#1}{#2}
}


\newcommand{\NBUA}{(\Sigma, Q, I, \Delta, F)}
\newcommand{\DBUA}{(\Sigma, Q, \init, \delta, F)}


\newcommand{\tiff}{\text{ if and only if }}


\documentclass{article}
\usepackage[utf8]{inputenc}
\usepackage[T1]{fontenc}

\usepackage{amsmath}
\usepackage{amssymb} 
\usepackage{amsthm}  
\usepackage{dsfont}
\usepackage{mathrsfs}
\usepackage{mathtools}

\usepackage{tikz-cd}
\usepackage{geometry}
\usepackage{hyperref}        

\usepackage[shortlabels]{enumitem}

\usepackage{fancyhdr}

\usepackage[hyperref,paper]{knowledge}

\usepackage{tre-uniformisation}

\knowledgeconfigure{label scope=false, notion, quotation, protect quotation={tikzcd, automata}}


\knowledge{tree}{notion}
\knowledge{Deterministic Bottom-Up Tree automaton}[DFTA]{notion}
\knowledge{Non-Deterministic Bottom-Up Tree automaton}[NFTA]{notion}
\knowledge{interpretation}{notion}
\knowledge{associated language}[lang]{notion}
\knowledge{nodes}{notion}
\knowledge{leaves}{notion}
\knowledge{label}{notion}
\knowledge{ancestor}{notion}
\knowledge{execution}[exec]{notion}
\knowledge{init}{notion}
\knowledge{order-invertible}{notion}


\begin{document}

\title{Uniformisation de MSO sur les arbres}

\maketitle

\section{Definitions}


\begin{definition}[Tree]
	A ""tree"" over an alphabet $\Sigma$ is recursively defined as follows:
	\begin{itemize}
		\item $a$, where $a \in \Sigma$.
		\item $a(t,t')$, where $a \in \Sigma$ and $t, t'$ are trees.
	\end{itemize}
\end{definition}

\begin{definition}[Nodes]
	The ""nodes"" of a $\tree$ are defined as follows:
	$$
		\nodes t = \left\{  \begin{array}{cc}
			\epsilon                                & \text{ si } t = a         \\
			\epsilon \cup \setdef {0u} {u \in \nodes {t'}}
			\cup \setdef {0u} {u \in \nodes {t''} } & \text{ si } t = a(t',t'')
		\end{array}
		\right.
	$$
\end{definition}


\begin{definition}[Leaves]
	The ""leaves"" of a $\tree$ are defined as follows:
	$$
		\leaves t = \left\{  \begin{array}{cc}
			\epsilon                                 & \text{ si } t = a         \\
			\setdef {0u} {u \in \leaves {t'}}
			\cup \setdef {0u} {u \in \leaves {t''} } & \text{ si } t = a(t',t'')
		\end{array}
		\right.
	$$
\end{definition}


\begin{notation}
	We note $\tlabel t n$ the ""label"" of the node $n$ in the tree $t$.
\end{notation}

\begin{definition}
	Let $t$ be a $\tree$ and $x, y \in \nodes t$, we define the ""ancestor"" relation as
	$$ \ancestor x y \iff \exists z \in \nodes t, y = xz$$
\end{definition}

\begin{definition}[""Deterministic Bottom-Up Tree automaton""]
	A Deterministic Bottom-Up Tree automaton (DFTA) is defined as a tuple:
	$$ (\Sigma, Q, \text{init} : \Sigma \to Q, \delta : \Sigma \times Q \times Q \to Q, F) $$
	where:
	\begin{itemize}
		\item $\Sigma$ is an alphabet.
		\item $Q$ is a finite set of states.
		\item ""init"" is a function that initializes the states of the leaves.
		\item $\delta$ is the transition function.
		\item $F$ is the set of final states.
	\end{itemize}
\end{definition}

\begin{definition}[""Non-Deterministic Bottom-Up Tree automaton""]
	A Non-Deterministic Bottom-Up Tree automaton (NFTA) is defined as a tuple:
	$$ (\Sigma, Q, I \subseteq \Sigma \times Q, \Delta \subseteq Q \times Q \times \Sigma \times Q, F) $$
\end{definition}

\begin{definition}[Interpretation of an automaton]
	The ""interpretation"" of an automaton $A$ over an alphabet $\Sigma$ is defined as follows:
	\begin{eqnarray*}
		\interpret A : \tree_{\Sigma} &\to& Q \\
		\interpret A (a) &=& \text{init}_a \\
		\interpret A (a(t,t')) &=& \delta_a (\interpret A (t), \interpret A (t'))
	\end{eqnarray*}
    \todo{extend definition to NFTA ?}
\end{definition}

\begin{definition}[Language of an automaton]
	Let $A$ be a "DFTA". Its ""associated language"" is defined as:
	$$\lang A = \setdef {t \in \tree_{\Sigma}} {\interpret A (t) \in F}$$
\end{definition}


\begin{definition}[Execution of an automaton]
	The ""execution"" of an automaton $A$ over a $\tree$ $t$ is :

	\begin{eqnarray*}
		\exec A : \nodes t &\to & Q \\
		\exec A (n)  &=& \init_{\tlabel t n} \text{ si } n \in \leaves t \\
		\exec A (n)  &=& \delta_{\tlabel t n}(\exec A (n0), \exec A (n1)) \text{ si } n \in \nodes t \cap \leaves t
	\end{eqnarray*}

	We say that $\exec A$ is acceptant if $\exec A (\epsilon) \in F$.
    \todo{extend definition to NFTA}
\end{definition}

\begin{remark}
	Let $A$ be a "DFTA",
	$$ \lang A = \setdef {t \in \tree_{\Sigma}} {\exec A (\epsilon) \in F} $$
\end{remark}

\begin{proof}
	We will proof that for any $\tree \ t$, $\interpret A (\tlabel t {\epsilon}) = \exec A (\epsilon)$.
	The proof procedes by induction on the structure of trees.
	\begin{itemize}
		\item $t = a$. Then
		      $\interpret A (\tlabel t {\epsilon}) = \interpret A (a)=  \init_a = \init_{\tlabel t {\epsilon}} = \exec A (\epsilon)$
		\item $t = a(t',t'')$. Then
		      \begin{eqnarray*}
			      \exec A (\epsilon)  &=& \delta_{\tlabel t {\epsilon}}(\exec A (0), \exec A (1)) \\
			      &=& \delta_a (\exec A (0), \exec A (1)) \\
			      &=& \delta_a (\exec A (\epsilon_{t'}), \exec A (\epsilon_{t''})) \\
			      &=& \delta_a (\interpret A (t'), \interpret A (t'')) \\
			      &=& \interpret A (t)
		      \end{eqnarray*}
	\end{itemize}
	\todo{reread this proof carefully and discuss it with Thomas.}
\end{proof}

\begin{theorem}
	Let $N$ be a "NFTA", there exists a "DFTA", $D$, such that
	$$\lang N = \lang D$$
\end{theorem}

\begin{proof}
	We consider the automaton
	$$ D = (\Sigma, \parts Q, \text{init} : \Sigma \to \parts Q, \delta : \Sigma \times \parts Q \times \parts Q \to \parts Q, F' \subseteq \parts Q) $$
	where
	\begin{itemize}
		\item $\init (a) = \setdef {q \in Q} {(a,q) \in I}$
		\item $\delta (a, X, Y) = \setdef {q \in  Q} {(x,y,a,q) \in \Delta, x \in X, y \in Y}$
		\item $F' = \setdef {X \in \parts Q} {X \cap F \neq \emptyset}$
	\end{itemize}
	\todo{Finish proof}
\end{proof}


\begin{definition}
	We say that an automaton is ""order-invertible"" if
	$$ \forall p,q,a,\, \delta_a (p,q) = \delta_a (q,p) $$ et
	and for the non deterministic ones
	$$ \forall p,q,a,r,\, (p,q,a,r) \in \Delta \iff  (q,p,a,r) \in \Delta $$

    \todo{discuss this name with Thomas}
\end{definition}


\begin{lemma}
	If $N$ a "NFTA" is "order-invertible", then it's corresponding deterministic automaton is also "order-invertible".
\end{lemma}

\begin{proof}
	Let $D$ be the deterministic automaton corresponding to $N$.
	\begin{eqnarray*}
		\delta_a (X,Y) &=& \setdef {q \in Q} {(x,y,a,q) \in \Delta, x \in X, y \in Y} \\
		&=& \setdef {q \in Q} {(y,x,a,q) \in \Delta, x \in X, y \in Y}  \reason {$N$ is "order-invertible"} \\
		&=& \delta_a (Y,X)
	\end{eqnarray*}
\end{proof}

\begin{theorem}
	If $A$ is a "order-invertible" "NFTA"  then there exists an MSO formula $\phi$ such that for all $\tree_{\Sigma} \ t$,
	$$ M(t) \models \phi \iff t \in \lang A$$
	\todo{Write up all of the missing definitions}
\end{theorem}

\begin{proof}
	\todo {}

\end{proof}



\bibliographystyle{alpha}
\bibliography{tre-uniformisation}

\end{document}


\begin{document}

\title{MSO uniformisation over trees}

\maketitle

\section{Trees}


\begin{definition}[Tree]
	A ""tree"" over an alphabet $\Sigma$ is recursively defined as follows:
	\begin{itemize}
		\item $a$, where $a \in \Sigma$.
		\item $a(t,t')$, where $a \in \Sigma$ and $t, t'$ are "trees".
	\end{itemize}

	The set of all "trees" over $\Sigma$ is $\tree_{\Sigma}$
\end{definition}

\begin{definition}[Nodes]
	The ""nodes"" of a "tree" are defined as follows:
	\[
		\intro* \Nodes t = \begin{cases}
			\epsilon                                & \text{ if } t = a         \\
			\epsilon \cup \setdef {0u} {u \in \Nodes {t'}}
			\cup \setdef {1u} {u \in \Nodes {t''} } & \text{ if } t = a(t',t'')
		\end{cases}
	\]
\end{definition}


\begin{definition}[Leaves]
	The ""leaves"" of a "tree" are defined as follows:
	\[
		\intro* \Leaves t = \begin{cases}
			\epsilon                                 & \text{ if } t = a         \\
			\setdef {0u} {u \in \Leaves {t'}}
			\cup \setdef {1u} {u \in \Leaves {t''} } & \text{ if } t = a(t',t'')
		\end{cases}
	\]
\end{definition}

\begin{remark}
	$\Leaves t \subseteq \Nodes t$
\end{remark}

\begin{definition}
	We note $\intro* \tlabel t n$ the ""label"" of the node $n$ in the "tree" $t$ :
	\[
		\intro* \tlabel t n =   \begin{cases}
			a               & \text{ if } t = a \text{ and } n = \epsilon         \\
			a               & \text{ if } t = a(t',t'') \text{ and } n = \epsilon \\
			\tlabel {t'} m  & \text{ if } t = a(t',t'') \text{ and } n = 0m       \\
			\tlabel {t''} m & \text{ if } t = a(t',t'') \text{ and } n = 1m
		\end{cases}
	\]
\end{definition}

\begin{definition}
	Let $t$ be a "tree" and $x, y \in \Nodes t$, we define the ""ancestor"" relation as
	\[  x \intro* \ancestor y \tiff \exists z \in \Nodes t, y = xz \]
\end{definition}

\section{Tree Automata}

\begin{definition}[""Non-Deterministic Bottom-Up Tree automaton""]
	A Non-Deterministic Bottom-Up Tree automaton (NBUA) is defined as a tuple
	$(\Sigma, Q, I, \Delta, F)$ where:
	\begin{itemize}
		\item $\Sigma$ is an alphabet.
		\item $Q$ is a finite set of ""states"".
		\item $I \subseteq \Sigma \times Q$ correspond to the possible states of the "leaves".
		\item $\Delta \subseteq Q \times Q \times \Sigma \times Q$ is the ""transition relation"".
		\item $F \subseteq Q$ is the set of ""final states"".
	\end{itemize}
\end{definition}

\begin{definition}[Run of an automaton]
	A ""run"" $\intro* \exec$ of a "NBUA" $A$ over a "tree" $t$ is :

	\begin{eqnarray*}
		\exec : \Nodes t &\to& Q \\
		(\tlabel t b,  \exec (n) ) \in I &\text{ if }& n \in \Leaves t \\
		(\exec (n0), \exec (n1), \tlabel t n, \exec (n)) \in \Delta &\text{ if }& n \in \Nodes t \setminus \Leaves t
	\end{eqnarray*}

	We say that $\exec$ is ""acceptant"" if $\exec (\epsilon) \in F$ and $A$ "accepts" $t$ if
	$F \cap \setdef { q\in Q} {\exec (\epsilon) = q} \neq \emptyset$
\end{definition}


\begin{definition}[Language of an automaton]
	Let $A$ be a "NBUA". Its ""associated language"" is defined as:
    \[\intro* \lang A = \setdef {t \in \tree_{\Sigma}} {\text{exists $\rho$ a "accepting" "run" of $A$ over $t$}} \]
\end{definition}


\begin{definition}[""Deterministic Bottom-Up Tree automaton""]
	A Deterministic Bottom-Up Tree automaton (DBUA) is "NBUA" that that verifies
	\[ \forall q,r,a,t,t' \ (q,r,a,t) \in \Delta \land   (q,r,a,t') \in \Delta  \implies t = t' \]
\end{definition}

\begin{remark}
	A "DBUA" can also be defined as as a tuple $(\Sigma, Q, \text{init}, \delta, F)$ where:
	\begin{itemize}
		\item $\Sigma$ is an alphabet.
		\item $Q$ is a finite set of "states".
		\item $\intro *\init : \Sigma \to Q$ is a function that initializes the states of the "leaves".
		\item $\delta : Q \times Q \times \Sigma \to Q$ is the ""transition map"".
		\item $F \subseteq Q$ is the set of "final states".
	\end{itemize}
	We will show that the two definitions are equivalent.
\end{remark}

\begin{definition}[Interpretation of an automaton]
	The ""interpretation"" of a "DBUA" $A$  over an alphabet $\Sigma$ is defined as follows:
	\begin{eqnarray*}
		\intro* \interpret A ~: \tree_{\Sigma} &\to& Q \\
		a &\mapsto& \init_a \\
		a(t,t') &\mapsto& \delta_a (\interpret A t, \interpret A {t'})
	\end{eqnarray*}
\end{definition}

\begin{remark}
	In the case of a "DBUA" there exists a unique "run" that corresponds to :

	\begin{eqnarray*}
		\exec : \Nodes t &\to & Q \\
		n  &\mapsto& \init_{\tlabel t n} \text{ if } n \in \Leaves t \\
		n  &\mapsto& \delta_{\tlabel t n}(\exec (n0), \exec (n1)) \text{ if } n \in \Nodes t \setminus \Leaves t
	\end{eqnarray*}
\end{remark}

\begin{lemma}
	Let $A$ be a "DBUA",
	\[ \lang A = \setdef {t \in \tree_{\Sigma}} {\interpret A t \in F} \]
\end{lemma}

\begin{proof}
	We will prove by induction on a "tree"~$t$ that (a) there exists a "run" of~$A$ over~$t$, and (b) for all "runs"~$\exec$ of $A$ over $t$, $\interpret A t = \exec (\epsilon)$.
	We proceed by case distinction.
	\begin{itemize}
		\item If $t = a$. Let $\rho$ be defined by $\rho(\varepsilon)=\init_a$, then $\rho$ is a "run" of~$A$ over~$t$. We have proved (a).
		      Consider now some "run" $\rho$ of $A$ over~$t$. We have $\rho(\epsilon)=\init_a=\interpret A a=\interpret A t$. We have proved  (b).

		\item If $t = a(t',t'')$. By induction hypothesis (a), there exists "runs"~$\rho'$ and $\rho''$ on~$t'$ and $t''$ respectively.
		      Let~$\rho$ be defined by $\rho(\epsilon)=\delta_{t(\epsilon)}(\rho'(\epsilon),\rho''(\epsilon))$,
		      $\rho(0u)=\rho'(u)$ for all $u\in\Nodes{t'}$ and $\rho(1u)=\rho''(u)$ for all~$u\in \Nodes{t''}$. Then $\rho$ is a "run" of~$A$ over $t$. We have proved (a).

		      Consider now some "run" $\rho$ of $A$ over $t$. Let~$\rho'(u)=\rho(0u)$ for all $u \in \Nodes {t'}$ and $\rho''(u)=\rho(0u)$ for all $u \in \Nodes {t''}$.
		      Then $\rho'$ is a "run" of~$A$ over $t'$ and $\rho''$ over $t''$. By induction hypothesis (b) twice, we know that $\rho'(\epsilon)=\interpret A {t'}$ and $\rho''(\epsilon)=\interpret A {t''}$.
		      We have now $\rho(\epsilon) = \delta_{t(\epsilon)}(\rho'(\epsilon),\rho''(\epsilon)) = \delta_{t(\epsilon)}(\interpret A {t'},\interpret A {t''}) = \interpret A t$. We have proved (b).
	\end{itemize}
\end{proof}

\begin{definition}
	Let $A = \NBUA$ be a "NBUA", we define the ""power set automaton"" of $A$, noted $\intro* \Det A$ as
	$\Det A = (\Sigma, \parts Q, \text{init}, \delta, F')$ where:

	\begin{itemize}
		\item $\begin{aligned}[t]
				      \init      : \Sigma & \to \parts Q                          \\
				      a                   & \mapsto \setdef{q \in Q}{(a,q) \in I}
			      \end{aligned} $

		\item $\begin{aligned}[t]
				      \delta               : \parts Q \times \parts Q \times \Sigma & \to \parts Q                                                     \\
				      (X, Y, a)                                                     & \mapsto \setdef{q \in Q}{(x,y,a,q) \in \Delta, x \in X, y \in Y}
			      \end{aligned}$

		\item $F' = \setdef{X \in \parts Q}{X \cap F \neq \emptyset} \subseteq \parts Q$
	\end{itemize}
\end{definition}

\begin{theorem}
	For all $N = \NBUA$ an "NBUA", the "power set automaton" verifies that
	\[
		\interpret {\Det N} t = \setdef {\rho(\epsilon)}{\text{$\rho$ "run" of~$N$ over~$t$}} \text{ and } \lang N = \lang {\Det N}
	\]
\end{theorem}

\begin{proof}
	We will first show that for all "tree" $t$,
	\[
		\interpret {\Det N} t = \setdef {\rho(\epsilon)}{\text{$\rho$ "run" of~$N$ over~$t$}}\ .
	\]
	We will prove this property by induction over $t$. We proceed by case distinction.
	\begin{itemize}
		\item If $t = a$, then
		      \begin{eqnarray*}
			      \setdef {\rho(\epsilon)}{\text{$\rho$ "run" of~$N$ over~$t$}} &=&  \setdef {q \in Q} {(a,q) \in I} \\
			      &=& \interpret {\Det N} t\ .
		      \end{eqnarray*}
		\item If $t = a(t_1,t_2)$, then
		      by the induction hypothesis we know that
		      $\interpret {\Det N} {t_i} = q_i$ \tiff there exists a "run" over $t_i$ ending at $q_i$, for $i \in \set {1,2}$.

		      \begin{eqnarray*}
			      &&q \in \interpret {\Det N} t \\
			      &\text{if and only if}& q \in \delta_a (\interpret {\Det N} {t_1}, \interpret {\Det N} {t_2}) \\
			      &\text{if and only if}& q \in \setdef {q \in  Q} {(x_1,x_2,a,q) \in \Delta, x_1 \in \interpret {\Det N} {t_1}, x_2 \in \interpret {\Det N} {t_2}} \\
			      &\text{if and only if}& \exists x_1,x_2, (x_1,x_2,a,q) \in \Delta, x_1 \in \interpret {\Det N} {t_1}, x_2 \in \interpret {\Det N} {t_2} \\
			      &\text{if and only if}& \text{there exists a "run" over }  t_i  \text { ending at } x_i \text { for  } i \in \set {1,2} \text{ and } (x_1,x_2,a,q) \in \Delta\\
			      &\text{if and only if}& q \in \setdef {\rho(\epsilon)}{\text{$\rho$ "run" of~$N$ over~$t$}}
		      \end{eqnarray*}
	\end{itemize}
	Then $ \interpret {\Det N} t = \setdef {\rho(\epsilon)}{\text{$\rho$ "run" of~$N$ over~$t$}}$.

	We must now show that $\Det N$ "accepts" $t$ \tiff $N$ "accepts" $t$.

	\begin{eqnarray*}
		N \text{ "accepts" } t &\tiff& F \cap \setdef {q \in Q} {(a,q) \in I} \neq \emptyset \\
		&\tiff&   F \cap  \interpret {\Det N} t \neq \emptyset \\
		&\tiff& \interpret {\Det N} t  \in F' \\
		&\tiff& {\Det N} \text{ "accepts" } t
	\end{eqnarray*}

	and so $\lang N = \lang {\Det N}$.

	\todo{Discuss proof}
\end{proof}


\begin{definition}
	An "NBUA" of "transition relation"~$\Delta$ is ""order-insensitive"" if for all "transitions@@relation" $(p,q,a,r) \in \Delta$,
	$(q,p,a,r) \in \Delta$.
\end{definition}

\begin{remark}
	For a "DBUA" of "transition map"~$\delta$, it is "order-insensitive" if and only if
	$\delta_a (p,q) = \delta_a (q,p)$ for all "states"~$p,q$ and letter~$a$.
\end{remark}

\begin{lemma}
	If $N = \NBUA$ a "NBUA" is "order-insensitive", then $\Det N$ is also "order-insensitive".
\end{lemma}

\begin{proof}
	\begin{eqnarray*}
		\delta_a (X,Y) &=& \setdef {q \in Q} {(x,y,a,q) \in \Delta, x \in X, y \in Y} \\
		&=& \setdef {q \in Q} {(y,x,a,q) \in \Delta, x \in X, y \in Y}  \reason {$N$ is "order-insensitive"} \\
		&=& \delta_a (Y,X)
	\end{eqnarray*}
\end{proof}

\subsection{MSO}

\begin{definition}
	Let $t \in \tree_{\Sigma}$, we define a model of $t$ as :

	\[\intro* \Model t := (\Nodes t, \ancestor , (a (x))_{a \in \Sigma}) \]

	where : $a(x) \tiff \tlabel t x = a$
\end{definition}

We will discuss some examples of MSO formulas.

\begin{example}
	The formula $\intro* \leaf (x) = \forall y,  y \ancestor x$ encodes the fact that $x$ is a "leaf".
\end{example}

\begin{example}
	$x$ is the root of the "tree" is equivalent to :
	\[\intro* \treeroot (x) = \forall y,  x \ancestor y \]
\end{example}

\begin{example}
	The following formula encodes the fact that $x$ and $y$ are the immediate children of $z$:
	\[\intro* \children (x,y,z) =
		x \neq y \land
		z  \ancestor  x \land  z \ancestor y \land
		\forall w
		\left( \left(
			z \ancestor w \land
				(w \ancestor x \lor w \ancestor  y) \right) \rightarrow (
			w = x \lor w = y
			)
		\right) \]
\end{example}

\begin{lemma}
	If $A$ is a "order-insensitive" "NBUA"  then there exists a MSO formula $\phi$ such that for all $t \in \tree_{\Sigma}$,
	\[ \Model t \models \phi \tiff t \in \lang A \]
\end{lemma}

\begin{proof}
	Let $A = \NBUA$ be a "NBUA", and $k = \abs Q$.
	\begin{eqnarray}
		\phi = \exists X_0 \ldots \exists X_{k-1} &\ & \forall x \left( \bigvee_{i=0}^{k-1} X_i(x) \right) \label{lem:AMSO1} \\
		&\land& \forall x \left( \bigwedge_{0 \leq i \leq j \leq k-1}  \lnot (X_i(x) \land X_j(x)) \right) \label{lem:AMSO2} \\
		&\land& \forall x  \left( \leaf (x) \ra \bigvee_{(a,i)\in I}  X_i(x) \right) \label{lem:AMSO3} \\
		&\land& \forall x \forall y \forall z \left(\children(x,y,z) \ra \bigvee_{(p,q,a,r)\in \Delta} \left(X_p(x) \land X_q(y) \land a(z) \land X_r(z)\right)\right) \label{lem:AMSO4} \\
		&\land& \forall x \left(\treeroot (x) \ra \bigvee_{q \in F} X_q(x) \right) \label{lem:AMSO5}
	\end{eqnarray}

	\todo{fix overfull}

	$j \in X_i$ \tiff, for a given "run", the state $i$ is reached from the node $j$.

	The formula $\phi$ encodes the following properties:

	\begin{enumerate}
		\item Every "node" has an associated "state".
		\item A node cannot be in two "states" at the same time.
		\item The "transitions@@relation" originate from the "leaves".
		\item The "transitions@@relation" of inner nodes ensure that for every trio of nodes, where two are children of the third, there exists a transition from the children to the parent.
		\item Finally, the word is accepted if and only if the state associated with the root node is a "final state".
	\end{enumerate}

	From this we deduce that
	\[ \Model t \models \phi \tiff t \in \lang A \]
\end{proof}

\begin{definition}
	The ""cartesian product automata"" of two "NBUA" $A_1 = (\Sigma, Q_1, I_1, \Delta_1, F_1)$ and $A_1 = (\Sigma, Q_2, I_2, \Delta_2, F_2)$ with "final states" $F' \subseteq Q_1 \times Q_2$ is :
	\[ A_1 \intro* \Atimes A_2 = (\Sigma, Q_1 \times Q_2, I_1 \times I_2, \Delta_3 , F') \]
	where $\Delta_3 = \setdef {((q_1,q_2) , (p_1,p_2), (a_1,a_2), (r_1,r_2)}    { (q_1,p_1,a_1,r_1) \in \Delta_1, (q_2,p_2,a_2,r_2) \in \Delta_2}$
\end{definition}

\begin{lemma}
	Let $A_1 = (\Sigma, Q_1, I_1, \Delta_1, F_1)$ and $A_1 = (\Sigma, Q_2, I_2, \Delta_2, F_2)$ be two "NBUA", then $A_3 = A_1 \Atimes A_2$ with "final states" $F_1 \times F_2$
	verifies that $\lang {A_3} = \lang {A_1} \cap \lang {A_2}$
\end{lemma}

\begin{lemma}
	Let $A_1 = (\Sigma, Q_1, I_1, \Delta_1, F_1)$ and $A_1 = (\Sigma, Q_2, I_2, \Delta_2, F_2)$ be two "NBUA", then
	\[ A_3 = (\Sigma, Q_1 \uplus Q_2, I_1 \uplus I_2, \Delta_1 \uplus \Delta_2, F_1 \uplus F_2) \]
	verifies that $\lang {A_3} = \lang {A_1} \cup \lang {A_2}$
\end{lemma}



\begin{lemma}
	Let $A = \NBUA$ the "NBUA" $A' = (\Sigma, Q, I, \Delta, Q \cap F)$
	verifies that $\lang {A'} = (\lang A )^{\complement}$
\end{lemma}


\iffalse
	\bibliographystyle{alpha}
	\bibliography{tre-uniformisation}
\fi

\end{document}
